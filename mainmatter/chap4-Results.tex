% Chapter 4: Results
\section{Binary outcome setting}
\subsection{Note on subsetted simulations}
Simulations with low observed event rates in the binary outcome model could result in large estimated treatment effects and standard errors, often corresponding to cases where the \texttt{glm()} algorithm did not converge.
For this reason, subsetted results are presented for simulation outcomes where both the \texttt{glm()} algorithm converged and at least one event was observed in each treatment arm.
Subsetting on these two criterion led to excluding greater than 30 percent of simulations in low sample size scenarios.
This issue occurs more frequently when sample size is small (n=32), outcome prevalence $(Pr(Y))$ is small, and for small treatment and prognostic factor effect sizes ($bZ$ and $bX$, respectively). 

\subsection{Binary predictors}
Power is substantially higher using CAA followed by re-randomization when compared to all other methods, including CAA followed by model-based inference.
The modest increase in power is especially apparent in low sample size settings and low marginal outcome prevalence, although the increased power is in part due to the inflated Type 1 error (see Table \ref{tab:b1p} rows 1 and 2).
The size issues with re-randomization occur only in specific settings, particularly where convergence issues with logistic regression are present.
Not all gains in power for CAA with re-randomization are due to Type 1 error inflation: when outcome prevalence increases to 50\% from 10\%, the type 1 error rate is 0.057 and 0.056 and approaches the nominal rate (see Table \ref{tab:b1p} Rows 13, 14) while power in all considered non-null effect sizes is consistently greater than the other methods.

Adjusting for prognostic factors used in the balancing method consistently increased power relative to unadjusted methods in almost all considered settings.
Evidence that unadjusted model-based regression approaches are too conservative is seen when comparing size of the tests for the binary outcome setting and the continuous outcome setting when using SBR with high prognostic factor effect size.
For example, see Table \ref{tab:b2p} for n=96, $(Pr(Y))=0.1$, $bZ=1$ and $bX=3$ for complete randomization: size of the unadjusted test is .032 and adjusted is .065. 
Similarly see Table \ref{tab:b3p} for continuous outcomes where n=96, $(Pr(X))=0.25$, $bZ=1$ and $bX=3$, size for SBR is .011 unadjusted an .047 after adjusting. 
Comparing model-based approaches, adjustment for prognostic factors was necessary to maintain proper Type 1 error control and to maximize power under almost all considered settings and outcome types.
The power advantage of adjusted relative to unadjusted analysis generally increases with prognostic factor effect sizes. 


The power benefit from using CAA followed by re-randomization becomes diminished relative to all other allocation procedures when sample size increases and marginal outcome prevalence increases (see Figure \ref{fig:b1p}, bottom right quadrant).
Even at large sample sizes re-randomization has greater power than model-based approaches, particularly when outcome prevalence is low.

Median bias (Table \ref{tab:b1mb}) is appreciably low and approaches zero when the estimated treatment effect is null or when trial size is large (n=96).
The bias results from glm() algorithm returning inflated estimates when no valid odds ratio estimate exists, \textit{i.e.} when the denominator of the odds ratio is zero because there are no events in the untreated arm.
This happens more often when considering within-strata odds ratios, which is why the median bias for adjusted estimates, though still small, is larger on average when compared to unadjusted estimates.

Coverage probability nears 100 percent at low outcome prevalence and low trial size, likely from standard errors being too large with low event rates.
Coverage probability nears 95 percent when outcome prevalence is high and when sample size is large (n=96).

Recall that subsetted results included only simulations where the algorithm converged and at least one event was observed in each treatment arm.
Compared to results on all simulations, subsetted power results show better type I error control for CAA followed by re-randomization overall (compare Table \ref{tab:b1p} rows 1,2 with Table \ref{tab:b1sp}, rows 1,2).
The modest power increase of CAA with re-randomization relative to other allocation procedures using model-based analysis is consistent in large and small sample sizes and high and low outcome prevalences.


\subsection{Continuous predictors}
Both CAA and SBR with model-based inference have higher power but fail to control size.
In low outcome prevalence (Pr(Y)=0.1) under strong prognostic factor effect size (bX=3) with a type I error threshold of 0.05, the power under the null setting is greater than 0.10 (see Table \ref{tab:b2p}, rows 1,2). 
As discussed in the previous section, the tendency for \texttt{glm()} to return large estimates and standard errors given certain simulation outcomes results in spuriously small p-values. 
Looking at the subsetted results in Table \ref{tab:b2sp} where size is at least as well controlled for re-randomization as other methods, power is highest with re-randomization or at least comparable to model-based approaches that adjust for prognostic factors.

Considering model-based approaches, adjusted estimates are consistently more powerful than unadjusted estimates. 
Adjusting for prognostic factors used in the balancing method controls median bias.
Median bias tended to be less than 10 percent and decreased with increasing sample size.

The coverage probability at low sample sizes was inflated due to low event rates, as described in the previous section on binary predictors.
All adjusted methods have coverage probabilities near 95 percent in the large sample size setting. 

\section{Continuous outcome setting}
\subsection{Binary predictors}
All methods perform comparably with respect to power. 
There is not as much of an advantage to re-randomization compared to model-based analysis approaches in the continuous outcome setting.

Median bias was tiny and coverage probability approached 95 percent for all methods and all conditions.
An exception is when prognostic factor effect size is large, stratified block randomization was used, and covariates were not adjusted for.

\subsection{Continuous predictors}
Adjusting for prognostic factors did not considerably increase power, except when prognostic factor effect size is large.
As in all simulated conditions with a continuous outcome, re-randomization following allocation does not confer an additional advantage in terms of increased power.